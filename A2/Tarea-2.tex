% Options for packages loaded elsewhere
\PassOptionsToPackage{unicode}{hyperref}
\PassOptionsToPackage{hyphens}{url}
%
\documentclass[
]{article}
\usepackage{amsmath,amssymb}
\usepackage{lmodern}
\usepackage{ifxetex,ifluatex}
\ifnum 0\ifxetex 1\fi\ifluatex 1\fi=0 % if pdftex
  \usepackage[T1]{fontenc}
  \usepackage[utf8]{inputenc}
  \usepackage{textcomp} % provide euro and other symbols
\else % if luatex or xetex
  \usepackage{unicode-math}
  \defaultfontfeatures{Scale=MatchLowercase}
  \defaultfontfeatures[\rmfamily]{Ligatures=TeX,Scale=1}
\fi
% Use upquote if available, for straight quotes in verbatim environments
\IfFileExists{upquote.sty}{\usepackage{upquote}}{}
\IfFileExists{microtype.sty}{% use microtype if available
  \usepackage[]{microtype}
  \UseMicrotypeSet[protrusion]{basicmath} % disable protrusion for tt fonts
}{}
\makeatletter
\@ifundefined{KOMAClassName}{% if non-KOMA class
  \IfFileExists{parskip.sty}{%
    \usepackage{parskip}
  }{% else
    \setlength{\parindent}{0pt}
    \setlength{\parskip}{6pt plus 2pt minus 1pt}}
}{% if KOMA class
  \KOMAoptions{parskip=half}}
\makeatother
\usepackage{xcolor}
\IfFileExists{xurl.sty}{\usepackage{xurl}}{} % add URL line breaks if available
\IfFileExists{bookmark.sty}{\usepackage{bookmark}}{\usepackage{hyperref}}
\hypersetup{
  pdftitle={Tarea\_2},
  pdfauthor={José Luis Jurado Ortiz},
  hidelinks,
  pdfcreator={LaTeX via pandoc}}
\urlstyle{same} % disable monospaced font for URLs
\usepackage[margin=1in]{geometry}
\usepackage{color}
\usepackage{fancyvrb}
\newcommand{\VerbBar}{|}
\newcommand{\VERB}{\Verb[commandchars=\\\{\}]}
\DefineVerbatimEnvironment{Highlighting}{Verbatim}{commandchars=\\\{\}}
% Add ',fontsize=\small' for more characters per line
\usepackage{framed}
\definecolor{shadecolor}{RGB}{248,248,248}
\newenvironment{Shaded}{\begin{snugshade}}{\end{snugshade}}
\newcommand{\AlertTok}[1]{\textcolor[rgb]{0.94,0.16,0.16}{#1}}
\newcommand{\AnnotationTok}[1]{\textcolor[rgb]{0.56,0.35,0.01}{\textbf{\textit{#1}}}}
\newcommand{\AttributeTok}[1]{\textcolor[rgb]{0.77,0.63,0.00}{#1}}
\newcommand{\BaseNTok}[1]{\textcolor[rgb]{0.00,0.00,0.81}{#1}}
\newcommand{\BuiltInTok}[1]{#1}
\newcommand{\CharTok}[1]{\textcolor[rgb]{0.31,0.60,0.02}{#1}}
\newcommand{\CommentTok}[1]{\textcolor[rgb]{0.56,0.35,0.01}{\textit{#1}}}
\newcommand{\CommentVarTok}[1]{\textcolor[rgb]{0.56,0.35,0.01}{\textbf{\textit{#1}}}}
\newcommand{\ConstantTok}[1]{\textcolor[rgb]{0.00,0.00,0.00}{#1}}
\newcommand{\ControlFlowTok}[1]{\textcolor[rgb]{0.13,0.29,0.53}{\textbf{#1}}}
\newcommand{\DataTypeTok}[1]{\textcolor[rgb]{0.13,0.29,0.53}{#1}}
\newcommand{\DecValTok}[1]{\textcolor[rgb]{0.00,0.00,0.81}{#1}}
\newcommand{\DocumentationTok}[1]{\textcolor[rgb]{0.56,0.35,0.01}{\textbf{\textit{#1}}}}
\newcommand{\ErrorTok}[1]{\textcolor[rgb]{0.64,0.00,0.00}{\textbf{#1}}}
\newcommand{\ExtensionTok}[1]{#1}
\newcommand{\FloatTok}[1]{\textcolor[rgb]{0.00,0.00,0.81}{#1}}
\newcommand{\FunctionTok}[1]{\textcolor[rgb]{0.00,0.00,0.00}{#1}}
\newcommand{\ImportTok}[1]{#1}
\newcommand{\InformationTok}[1]{\textcolor[rgb]{0.56,0.35,0.01}{\textbf{\textit{#1}}}}
\newcommand{\KeywordTok}[1]{\textcolor[rgb]{0.13,0.29,0.53}{\textbf{#1}}}
\newcommand{\NormalTok}[1]{#1}
\newcommand{\OperatorTok}[1]{\textcolor[rgb]{0.81,0.36,0.00}{\textbf{#1}}}
\newcommand{\OtherTok}[1]{\textcolor[rgb]{0.56,0.35,0.01}{#1}}
\newcommand{\PreprocessorTok}[1]{\textcolor[rgb]{0.56,0.35,0.01}{\textit{#1}}}
\newcommand{\RegionMarkerTok}[1]{#1}
\newcommand{\SpecialCharTok}[1]{\textcolor[rgb]{0.00,0.00,0.00}{#1}}
\newcommand{\SpecialStringTok}[1]{\textcolor[rgb]{0.31,0.60,0.02}{#1}}
\newcommand{\StringTok}[1]{\textcolor[rgb]{0.31,0.60,0.02}{#1}}
\newcommand{\VariableTok}[1]{\textcolor[rgb]{0.00,0.00,0.00}{#1}}
\newcommand{\VerbatimStringTok}[1]{\textcolor[rgb]{0.31,0.60,0.02}{#1}}
\newcommand{\WarningTok}[1]{\textcolor[rgb]{0.56,0.35,0.01}{\textbf{\textit{#1}}}}
\usepackage{graphicx}
\makeatletter
\def\maxwidth{\ifdim\Gin@nat@width>\linewidth\linewidth\else\Gin@nat@width\fi}
\def\maxheight{\ifdim\Gin@nat@height>\textheight\textheight\else\Gin@nat@height\fi}
\makeatother
% Scale images if necessary, so that they will not overflow the page
% margins by default, and it is still possible to overwrite the defaults
% using explicit options in \includegraphics[width, height, ...]{}
\setkeys{Gin}{width=\maxwidth,height=\maxheight,keepaspectratio}
% Set default figure placement to htbp
\makeatletter
\def\fps@figure{htbp}
\makeatother
\setlength{\emergencystretch}{3em} % prevent overfull lines
\providecommand{\tightlist}{%
  \setlength{\itemsep}{0pt}\setlength{\parskip}{0pt}}
\setcounter{secnumdepth}{-\maxdimen} % remove section numbering
\ifluatex
  \usepackage{selnolig}  % disable illegal ligatures
\fi

\title{Tarea\_2}
\author{José Luis Jurado Ortiz}
\date{19/9/2021}

\begin{document}
\maketitle

\hypertarget{instrucciones-preliminares}{%
\section{Instrucciones preliminares}\label{instrucciones-preliminares}}

\begin{itemize}
\tightlist
\item
  Empieza abriendo el proyecto de RStudio correspondiente a tu
  repositorio personal de la asignatura.
\item
  En todas las tareas tendrás que repetir un proceso como el descrito en
  la sección Repite los pasos creando un fichero Rmarkdown para esta
  práctica de la Práctica00. Puedes releer la sección Practicando la
  entrega de las Tareas de esa misma práctica para recordar el
  procedimiento de entrega.
\end{itemize}

\hypertarget{ejercicio-1.-simulando-variables-aleatorias-discretas.}{%
\section{Ejercicio 1. Simulando variables aleatorias
discretas.}\label{ejercicio-1.-simulando-variables-aleatorias-discretas.}}

\textbf{Apartado 1}: La variable aleatoria discreta \(X_1\) tiene esta
tabla de densidad de probabilidad (es la variable que se usa como
ejemplo en la Sesión ):

\begin{table}[]
\centering
\begin{tabular}
valor $X_1$ & 0 & 1 & 2 & 3 \\
Probabilidad de ese valor *$P(X=x_i)$*&$\frac{64}{125}$&$\frac{48}{125}$&$\frac{12}{125}$&$\frac{1}{125}$
\end{tabular}
\end{table}

Para el cálculo de la media, se utiliza la media poblacional teórica ya
que sabemos la población total y la probabilidad de aparición de todos
los individuos de esta población. Se calculará con la fórmula:
\(\mu=\sum_1^nx_ip_i\)

\begin{Shaded}
\begin{Highlighting}[]
\NormalTok{(}\AttributeTok{media\_teorica =} \DecValTok{0}\SpecialCharTok{*}\DecValTok{64}\SpecialCharTok{/}\DecValTok{125}\SpecialCharTok{+}\DecValTok{1}\SpecialCharTok{*}\DecValTok{48}\SpecialCharTok{/}\DecValTok{125}\SpecialCharTok{+}\DecValTok{2}\SpecialCharTok{*}\DecValTok{12}\SpecialCharTok{/}\DecValTok{125}\SpecialCharTok{+}\DecValTok{3}\SpecialCharTok{*}\DecValTok{1}\SpecialCharTok{/}\DecValTok{125}\NormalTok{)}
\end{Highlighting}
\end{Shaded}

\begin{verbatim}
## [1] 0.6
\end{verbatim}

Para el cálculo de la varianza teórica, se usará la varianza poblacional
con la fórmula: \(\sigma^2=\sum_{1}^{n} (x_{i}-\mu)^2p_{i}\)

\begin{Shaded}
\begin{Highlighting}[]
\NormalTok{(}\AttributeTok{varianza\_teorica =}\NormalTok{ (}\DecValTok{0}\SpecialCharTok{{-}}\NormalTok{media\_teorica)}\SpecialCharTok{\^{}}\DecValTok{2}\SpecialCharTok{*}\DecValTok{64}\SpecialCharTok{/}\DecValTok{125}\SpecialCharTok{+}\NormalTok{(}\DecValTok{1}\SpecialCharTok{{-}}\NormalTok{media\_teorica)}\SpecialCharTok{\^{}}\DecValTok{2}\SpecialCharTok{*}\DecValTok{48}\SpecialCharTok{/}\DecValTok{125}\SpecialCharTok{+}\NormalTok{(}\DecValTok{2}\SpecialCharTok{{-}}\NormalTok{media\_teorica)}\SpecialCharTok{\^{}}\DecValTok{2}\SpecialCharTok{*}\DecValTok{12}\SpecialCharTok{/}\DecValTok{125}\SpecialCharTok{+}\NormalTok{(}\DecValTok{3}\SpecialCharTok{{-}}\NormalTok{media\_teorica)}\SpecialCharTok{\^{}}\DecValTok{2}\SpecialCharTok{*}\DecValTok{1}\SpecialCharTok{/}\DecValTok{125}\NormalTok{)}
\end{Highlighting}
\end{Shaded}

\begin{verbatim}
## [1] 0.48
\end{verbatim}

\hypertarget{apartado-2}{%
\subsection{Apartado 2}\label{apartado-2}}

Para realizar la simulación, tomamos cien mil muestras de tamaño 10 y
posteriormente realizamos la media de dichas muestras.

\begin{Shaded}
\begin{Highlighting}[]
\NormalTok{k }\OtherTok{=} \DecValTok{100000}

\NormalTok{mediasMuestrales }\OtherTok{=} \FunctionTok{replicate}\NormalTok{(k, \{ }
\NormalTok{  muestra }\OtherTok{=} \FunctionTok{sample}\NormalTok{(}\DecValTok{0}\SpecialCharTok{:}\DecValTok{3}\NormalTok{, }\AttributeTok{size=}\DecValTok{10}\NormalTok{, }\AttributeTok{replace=}\NormalTok{T, }\AttributeTok{prob =} \FunctionTok{c}\NormalTok{(}\DecValTok{64}\NormalTok{, }\DecValTok{48}\NormalTok{, }\DecValTok{12}\NormalTok{, }\DecValTok{1}\NormalTok{))}
  \FunctionTok{mean}\NormalTok{(muestra)}
\NormalTok{\})}

\FunctionTok{mean}\NormalTok{(mediasMuestrales)}
\end{Highlighting}
\end{Shaded}

\begin{verbatim}
## [1] 0.599991
\end{verbatim}

Las gráficas para ilustrar la simulación son el histograma y una línea
que representa la mediamuestral

\begin{Shaded}
\begin{Highlighting}[]
\FunctionTok{library}\NormalTok{(ggplot2)}
\FunctionTok{library}\NormalTok{(tidyverse)}
\end{Highlighting}
\end{Shaded}

\begin{verbatim}
## -- Attaching packages --------------------------------------- tidyverse 1.3.1 --
\end{verbatim}

\begin{verbatim}
## v tibble  3.1.4     v dplyr   1.0.7
## v tidyr   1.1.3     v stringr 1.4.0
## v readr   2.0.1     v forcats 0.5.1
## v purrr   0.3.4
\end{verbatim}

\begin{verbatim}
## -- Conflicts ------------------------------------------ tidyverse_conflicts() --
## x dplyr::filter() masks stats::filter()
## x dplyr::lag()    masks stats::lag()
\end{verbatim}

\begin{Shaded}
\begin{Highlighting}[]
\NormalTok{mediasMuestrales }\SpecialCharTok{\%\textgreater{}\%}
\NormalTok{  as\_tibble }\SpecialCharTok{\%\textgreater{}\%}
\FunctionTok{ggplot}\NormalTok{() }\SpecialCharTok{+}
  \FunctionTok{geom\_histogram}\NormalTok{(}\FunctionTok{aes}\NormalTok{(}\AttributeTok{x =}\NormalTok{ value), }\AttributeTok{bins =} \DecValTok{15}\NormalTok{, }\AttributeTok{fill=}\StringTok{"orange"}\NormalTok{, }\AttributeTok{color=}\StringTok{"black"}\NormalTok{) }\SpecialCharTok{+}
  \FunctionTok{geom\_vline}\NormalTok{(}\AttributeTok{xintercept =} \FunctionTok{mean}\NormalTok{(mediasMuestrales),}
             \AttributeTok{col=}\StringTok{"blue"}\NormalTok{, }\AttributeTok{linetype=}\StringTok{"dashed"}\NormalTok{, }\AttributeTok{size=}\FloatTok{1.5}\NormalTok{)}
\end{Highlighting}
\end{Shaded}

\includegraphics{Tarea-2_files/figure-latex/unnamed-chunk-4-1.pdf}

Cambiamos el tamaño de la muestra a 30.

\begin{Shaded}
\begin{Highlighting}[]
\NormalTok{k2 }\OtherTok{=} \DecValTok{100000}

\NormalTok{mediasMuestrales2 }\OtherTok{=} \FunctionTok{replicate}\NormalTok{(k2, \{ }
\NormalTok{  muestra }\OtherTok{=} \FunctionTok{sample}\NormalTok{(}\DecValTok{0}\SpecialCharTok{:}\DecValTok{3}\NormalTok{, }\AttributeTok{size=}\DecValTok{30}\NormalTok{, }\AttributeTok{replace=}\NormalTok{T, }\AttributeTok{prob =} \FunctionTok{c}\NormalTok{(}\DecValTok{64}\NormalTok{, }\DecValTok{48}\NormalTok{, }\DecValTok{12}\NormalTok{, }\DecValTok{1}\NormalTok{))}
  \FunctionTok{mean}\NormalTok{(muestra)}
\NormalTok{\})}

\FunctionTok{mean}\NormalTok{(mediasMuestrales2)}
\end{Highlighting}
\end{Shaded}

\begin{verbatim}
## [1] 0.600493
\end{verbatim}

\begin{Shaded}
\begin{Highlighting}[]
\FunctionTok{library}\NormalTok{(ggplot2)}
\FunctionTok{library}\NormalTok{(tidyverse)}

\NormalTok{mediasMuestrales2 }\SpecialCharTok{\%\textgreater{}\%}
\NormalTok{  as\_tibble }\SpecialCharTok{\%\textgreater{}\%}
\FunctionTok{ggplot}\NormalTok{() }\SpecialCharTok{+}
  \FunctionTok{geom\_histogram}\NormalTok{(}\FunctionTok{aes}\NormalTok{(}\AttributeTok{x =}\NormalTok{ value), }\AttributeTok{bins =} \DecValTok{20}\NormalTok{, }\AttributeTok{fill=}\StringTok{"orange"}\NormalTok{, }\AttributeTok{color=}\StringTok{"black"}\NormalTok{) }\SpecialCharTok{+}
  \FunctionTok{geom\_vline}\NormalTok{(}\AttributeTok{xintercept =} \FunctionTok{mean}\NormalTok{(mediasMuestrales),}
             \AttributeTok{col=}\StringTok{"blue"}\NormalTok{, }\AttributeTok{linetype=}\StringTok{"dashed"}\NormalTok{, }\AttributeTok{size=}\FloatTok{1.5}\NormalTok{)}
\end{Highlighting}
\end{Shaded}

\includegraphics{Tarea-2_files/figure-latex/unnamed-chunk-6-1.pdf}

\hypertarget{apartado-3}{%
\subsection{Apartado 3}\label{apartado-3}}

\begin{Shaded}
\begin{Highlighting}[]
\FunctionTok{library}\NormalTok{(tidyverse)}
\NormalTok{X1 }\OtherTok{\textless{}{-}} \FunctionTok{tibble}\NormalTok{(}\AttributeTok{valor =} \FunctionTok{c}\NormalTok{(}\DecValTok{0}\NormalTok{,}\DecValTok{1}\NormalTok{,}\DecValTok{2}\NormalTok{), }\AttributeTok{prob =} \FunctionTok{c}\NormalTok{(}\DecValTok{1}\SpecialCharTok{/}\DecValTok{2}\NormalTok{,}\DecValTok{1}\SpecialCharTok{/}\DecValTok{4}\NormalTok{,}\DecValTok{1}\SpecialCharTok{/}\DecValTok{4}\NormalTok{))}
\NormalTok{X2 }\OtherTok{\textless{}{-}} \FunctionTok{tibble}\NormalTok{(}\AttributeTok{valor =} \FunctionTok{c}\NormalTok{(}\DecValTok{0}\NormalTok{,}\DecValTok{1}\NormalTok{,}\DecValTok{2}\NormalTok{,}\DecValTok{3}\NormalTok{), }\AttributeTok{prob =} \FunctionTok{c}\NormalTok{(}\DecValTok{64}\SpecialCharTok{/}\DecValTok{125}\NormalTok{,}\DecValTok{48}\SpecialCharTok{/}\DecValTok{125}\NormalTok{,}\DecValTok{12}\SpecialCharTok{/}\DecValTok{125}\NormalTok{,}\DecValTok{1}\SpecialCharTok{/}\DecValTok{125}\NormalTok{))}

\NormalTok{sol }\OtherTok{\textless{}{-}} \FunctionTok{tibble}\NormalTok{(}\AttributeTok{valor =} \FunctionTok{c}\NormalTok{(}\DecValTok{0}\SpecialCharTok{:}\DecValTok{5}\NormalTok{), }\AttributeTok{prob=}\FunctionTok{c}\NormalTok{(}\DecValTok{1}\SpecialCharTok{/}\DecValTok{2}\SpecialCharTok{*}\DecValTok{64}\SpecialCharTok{/}\DecValTok{125}\NormalTok{,}\DecValTok{1}\SpecialCharTok{/}\DecValTok{2}\SpecialCharTok{*}\DecValTok{48}\SpecialCharTok{/}\DecValTok{125}\SpecialCharTok{+}\DecValTok{64}\SpecialCharTok{/}\DecValTok{125}\SpecialCharTok{*}\DecValTok{1}\SpecialCharTok{/}\DecValTok{4}\NormalTok{,}\DecValTok{1}\SpecialCharTok{/}\DecValTok{2}\SpecialCharTok{*}\DecValTok{12}\SpecialCharTok{/}\DecValTok{125}\SpecialCharTok{+}\DecValTok{64}\SpecialCharTok{/}\DecValTok{125}\SpecialCharTok{*}\DecValTok{1}\SpecialCharTok{/}\DecValTok{4}\SpecialCharTok{+}\DecValTok{1}\SpecialCharTok{/}\DecValTok{4}\SpecialCharTok{*}\DecValTok{48}\SpecialCharTok{/}\DecValTok{125}\NormalTok{,}\DecValTok{1}\SpecialCharTok{/}\DecValTok{4}\SpecialCharTok{*}\DecValTok{12}\SpecialCharTok{/}\DecValTok{125}\SpecialCharTok{+}\DecValTok{48}\SpecialCharTok{/}\DecValTok{125}\SpecialCharTok{*}\DecValTok{1}\SpecialCharTok{/}\DecValTok{4}\SpecialCharTok{+}\DecValTok{1}\SpecialCharTok{/}\DecValTok{2}\SpecialCharTok{*}\DecValTok{1}\SpecialCharTok{/}\DecValTok{125}\NormalTok{,}\DecValTok{1}\SpecialCharTok{/}\DecValTok{4}\SpecialCharTok{*}\DecValTok{1}\SpecialCharTok{/}\DecValTok{125}\SpecialCharTok{+}\DecValTok{1}\SpecialCharTok{/}\DecValTok{4}\SpecialCharTok{*}\DecValTok{12}\SpecialCharTok{/}\DecValTok{125}\NormalTok{,}\DecValTok{1}\SpecialCharTok{/}\DecValTok{4}\SpecialCharTok{*}\DecValTok{1}\SpecialCharTok{/}\DecValTok{125}\NormalTok{))}
\end{Highlighting}
\end{Shaded}

\#\#Apartado 4

\begin{Shaded}
\begin{Highlighting}[]
\NormalTok{mediaX1 }\OtherTok{\textless{}{-}}\NormalTok{ X1 }\SpecialCharTok{\%\textgreater{}\%} 
  \FunctionTok{mutate}\NormalTok{(}\AttributeTok{producto =}\NormalTok{ valor}\SpecialCharTok{*}\NormalTok{prob) }\SpecialCharTok{\%\textgreater{}\%} 
  \FunctionTok{summarise}\NormalTok{(}\AttributeTok{media =} \FunctionTok{sum}\NormalTok{(producto))}
\NormalTok{mediaX1}
\end{Highlighting}
\end{Shaded}

\begin{verbatim}
## # A tibble: 1 x 1
##   media
##   <dbl>
## 1  0.75
\end{verbatim}

\begin{Shaded}
\begin{Highlighting}[]
\NormalTok{mediaX2 }\OtherTok{\textless{}{-}}\NormalTok{ X2 }\SpecialCharTok{\%\textgreater{}\%} 
  \FunctionTok{mutate}\NormalTok{(}\AttributeTok{producto =}\NormalTok{ valor}\SpecialCharTok{*}\NormalTok{prob) }\SpecialCharTok{\%\textgreater{}\%} 
  \FunctionTok{summarise}\NormalTok{(}\AttributeTok{media =} \FunctionTok{sum}\NormalTok{(producto))}
\NormalTok{mediaX2}
\end{Highlighting}
\end{Shaded}

\begin{verbatim}
## # A tibble: 1 x 1
##   media
##   <dbl>
## 1   0.6
\end{verbatim}

\begin{Shaded}
\begin{Highlighting}[]
\NormalTok{(mediaTeorica }\OtherTok{\textless{}{-}}\NormalTok{ mediaX1}\SpecialCharTok{+}\NormalTok{mediaX2)}
\end{Highlighting}
\end{Shaded}

\begin{verbatim}
##   media
## 1  1.35
\end{verbatim}

\begin{Shaded}
\begin{Highlighting}[]
\NormalTok{k }\OtherTok{=} \DecValTok{100000}

\NormalTok{sumaValores }\OtherTok{=} \FunctionTok{replicate}\NormalTok{(k,\{}
\NormalTok{  primerValor }\OtherTok{=} \FunctionTok{sample}\NormalTok{(X1}\SpecialCharTok{$}\NormalTok{valor,}\AttributeTok{size =} \DecValTok{1}\NormalTok{,}\AttributeTok{prob =}\NormalTok{ X1}\SpecialCharTok{$}\NormalTok{prob)}
\NormalTok{  segundoValor }\OtherTok{=} \FunctionTok{sample}\NormalTok{(X2}\SpecialCharTok{$}\NormalTok{valor,}\AttributeTok{size =} \DecValTok{1}\NormalTok{ ,}\AttributeTok{prob =}\NormalTok{ X2}\SpecialCharTok{$}\NormalTok{prob)}
  
  \FunctionTok{return}\NormalTok{(primerValor}\SpecialCharTok{+}\NormalTok{segundoValor)}
\NormalTok{\})}

\FunctionTok{mean}\NormalTok{(sumaValores)}
\end{Highlighting}
\end{Shaded}

\begin{verbatim}
## [1] 1.3454
\end{verbatim}

\#Ejercicio 2

\begin{Shaded}
\begin{Highlighting}[]
\NormalTok{X }\OtherTok{\textless{}{-}} \FunctionTok{read\_csv}\NormalTok{(}\FunctionTok{url}\NormalTok{(}\StringTok{\textquotesingle{}https://gist.githubusercontent.com/fernandosansegundo/471b4887737cfcec7e9cf28631f2e21e/raw/b3944599d02df494f5903740db5acac9da35bc6f/testResults.csv\textquotesingle{}}\NormalTok{))}
\end{Highlighting}
\end{Shaded}

\begin{verbatim}
## Rows: 200 Columns: 9
\end{verbatim}

\begin{verbatim}
## -- Column specification --------------------------------------------------------
## Delimiter: ","
## chr (2): name, gender_age
## dbl (7): id, test_number, week1, week2, week3, week4, week5
\end{verbatim}

\begin{verbatim}
## 
## i Use `spec()` to retrieve the full column specification for this data.
## i Specify the column types or set `show_col_types = FALSE` to quiet this message.
\end{verbatim}

\begin{Shaded}
\begin{Highlighting}[]
\NormalTok{X}
\end{Highlighting}
\end{Shaded}

\begin{verbatim}
## # A tibble: 200 x 9
##    name           id gender_age test_number week1 week2 week3 week4 week5
##    <chr>       <dbl> <chr>            <dbl> <dbl> <dbl> <dbl> <dbl> <dbl>
##  1 Jacob         108 m_20                 1     8     5     7     5     6
##  2 Jacob         108 m_20                 2     2     2     4     0     3
##  3 Michael       490 m_19                 1    10     0     5     4     0
##  4 Michael       490 m_19                 2     9    10     8    10     9
##  5 Matthew       424 m_18                 1     6     0     0     1    10
##  6 Matthew       424 m_18                 2     3     4     2     5     8
##  7 Joshua        734 m_17                 1    10     2     2     0     6
##  8 Joshua        734 m_17                 2    10     0     6     8     9
##  9 Christopher   928 m_20                 1     5     2     0     0     0
## 10 Christopher   928 m_20                 2     9     9     3    10     4
## # ... with 190 more rows
\end{verbatim}

\begin{Shaded}
\begin{Highlighting}[]
\FunctionTok{library}\NormalTok{(tidyverse)}

\NormalTok{firstMod }\OtherTok{\textless{}{-}}\NormalTok{ X }\SpecialCharTok{\%\textgreater{}\%} 
  \FunctionTok{pivot\_wider}\NormalTok{(}\AttributeTok{names\_from =}\NormalTok{ test\_number,}\AttributeTok{values\_from =} \FunctionTok{c}\NormalTok{(week1,week2,week3,week4,week5))}

\NormalTok{firstMod}
\end{Highlighting}
\end{Shaded}

\begin{verbatim}
## # A tibble: 100 x 13
##    name           id gender_age week1_1 week1_2 week2_1 week2_2 week3_1 week3_2
##    <chr>       <dbl> <chr>        <dbl>   <dbl>   <dbl>   <dbl>   <dbl>   <dbl>
##  1 Jacob         108 m_20             8       2       5       2       7       4
##  2 Michael       490 m_19            10       9       0      10       5       8
##  3 Matthew       424 m_18             6       3       0       4       0       2
##  4 Joshua        734 m_17            10      10       2       0       2       6
##  5 Christopher   928 m_20             5       9       2       9       0       3
##  6 Nicholas      492 m_17             1       1       2       5       9       7
##  7 Andrew        678 m_17             4       6      10      10       6       4
##  8 Joseph        776 m_17             1       8       1       3       5       1
##  9 Daniel        549 m_20             5       2       9       0       8      10
## 10 Tyler         530 m_19             4       6       6      10       8       7
## # ... with 90 more rows, and 4 more variables: week4_1 <dbl>, week4_2 <dbl>,
## #   week5_1 <dbl>, week5_2 <dbl>
\end{verbatim}

\begin{Shaded}
\begin{Highlighting}[]
\NormalTok{secondMod }\OtherTok{\textless{}{-}}\NormalTok{ firstMod }\SpecialCharTok{\%\textgreater{}\%} 
  \FunctionTok{separate}\NormalTok{(gender\_age, }\AttributeTok{into =} \FunctionTok{c}\NormalTok{(}\StringTok{"gender"}\NormalTok{, }\StringTok{"age"}\NormalTok{))}

\NormalTok{secondMod}
\end{Highlighting}
\end{Shaded}

\begin{verbatim}
## # A tibble: 100 x 14
##    name           id gender age   week1_1 week1_2 week2_1 week2_2 week3_1 week3_2
##    <chr>       <dbl> <chr>  <chr>   <dbl>   <dbl>   <dbl>   <dbl>   <dbl>   <dbl>
##  1 Jacob         108 m      20          8       2       5       2       7       4
##  2 Michael       490 m      19         10       9       0      10       5       8
##  3 Matthew       424 m      18          6       3       0       4       0       2
##  4 Joshua        734 m      17         10      10       2       0       2       6
##  5 Christopher   928 m      20          5       9       2       9       0       3
##  6 Nicholas      492 m      17          1       1       2       5       9       7
##  7 Andrew        678 m      17          4       6      10      10       6       4
##  8 Joseph        776 m      17          1       8       1       3       5       1
##  9 Daniel        549 m      20          5       2       9       0       8      10
## 10 Tyler         530 m      19          4       6       6      10       8       7
## # ... with 90 more rows, and 4 more variables: week4_1 <dbl>, week4_2 <dbl>,
## #   week5_1 <dbl>, week5_2 <dbl>
\end{verbatim}

\#Ejercicio 3

\#\#Apartado 1

\begin{Shaded}
\begin{Highlighting}[]
\FunctionTok{library}\NormalTok{(tidyverse)}

\CommentTok{\#Código copiado de internet}
\NormalTok{panel.cor }\OtherTok{\textless{}{-}} \ControlFlowTok{function}\NormalTok{(x, y, }\AttributeTok{digits =} \DecValTok{2}\NormalTok{, }\AttributeTok{prefix =} \StringTok{""}\NormalTok{, cex.cor, ...) \{}
\NormalTok{    usr }\OtherTok{\textless{}{-}} \FunctionTok{par}\NormalTok{(}\StringTok{"usr"}\NormalTok{)}
    \FunctionTok{on.exit}\NormalTok{(}\FunctionTok{par}\NormalTok{(usr))}
    \FunctionTok{par}\NormalTok{(}\AttributeTok{usr =} \FunctionTok{c}\NormalTok{(}\DecValTok{0}\NormalTok{, }\DecValTok{1}\NormalTok{, }\DecValTok{0}\NormalTok{, }\DecValTok{1}\NormalTok{))}
\NormalTok{    Cor }\OtherTok{\textless{}{-}} \FunctionTok{abs}\NormalTok{(}\FunctionTok{cor}\NormalTok{(x, y))}
\NormalTok{    txt }\OtherTok{\textless{}{-}} \FunctionTok{paste0}\NormalTok{(prefix, }\FunctionTok{format}\NormalTok{(}\FunctionTok{c}\NormalTok{(Cor, }\FloatTok{0.123456789}\NormalTok{), }\AttributeTok{digits =}\NormalTok{ digits)[}\DecValTok{1}\NormalTok{])}
    \ControlFlowTok{if}\NormalTok{(}\FunctionTok{missing}\NormalTok{(cex.cor)) \{}
\NormalTok{        cex.cor }\OtherTok{\textless{}{-}} \FloatTok{0.4} \SpecialCharTok{/} \FunctionTok{strwidth}\NormalTok{(txt)}
\NormalTok{    \}}
    \FunctionTok{text}\NormalTok{(}\FloatTok{0.5}\NormalTok{, }\FloatTok{0.5}\NormalTok{, txt,}
         \AttributeTok{cex =} \DecValTok{1} \SpecialCharTok{+}\NormalTok{ cex.cor }\SpecialCharTok{*}\NormalTok{ Cor)}
\NormalTok{\}}


\FunctionTok{pairs}\NormalTok{(diamonds,}
      \AttributeTok{upper.panel =}\NormalTok{ panel.cor,}
      \AttributeTok{lower.panel =}\NormalTok{ panel.smooth)}
\end{Highlighting}
\end{Shaded}

\hypertarget{apartado-2-1}{%
\subsection{Apartado 2}\label{apartado-2-1}}

\begin{Shaded}
\begin{Highlighting}[]
\CommentTok{\#Código del libro}
\NormalTok{who1 }\OtherTok{\textless{}{-}}\NormalTok{ who }\SpecialCharTok{\%\textgreater{}\%}
  \FunctionTok{pivot\_longer}\NormalTok{(}
    \AttributeTok{cols =}\NormalTok{ new\_sp\_m014}\SpecialCharTok{:}\NormalTok{newrel\_f65, }
    \AttributeTok{names\_to =} \StringTok{"key"}\NormalTok{, }
    \AttributeTok{values\_to =} \StringTok{"cases"}\NormalTok{, }
    \AttributeTok{values\_drop\_na =} \ConstantTok{TRUE}
\NormalTok{  ) }\SpecialCharTok{\%\textgreater{}\%} 
  \FunctionTok{mutate}\NormalTok{(}
    \AttributeTok{key =}\NormalTok{ stringr}\SpecialCharTok{::}\FunctionTok{str\_replace}\NormalTok{(key, }\StringTok{"newrel"}\NormalTok{, }\StringTok{"new\_rel"}\NormalTok{)}
\NormalTok{  ) }\SpecialCharTok{\%\textgreater{}\%}
  \FunctionTok{separate}\NormalTok{(key, }\FunctionTok{c}\NormalTok{(}\StringTok{"new"}\NormalTok{, }\StringTok{"var"}\NormalTok{, }\StringTok{"sexage"}\NormalTok{)) }\SpecialCharTok{\%\textgreater{}\%} 
  \FunctionTok{select}\NormalTok{(}\SpecialCharTok{{-}}\NormalTok{new, }\SpecialCharTok{{-}}\NormalTok{iso2, }\SpecialCharTok{{-}}\NormalTok{iso3) }\SpecialCharTok{\%\textgreater{}\%} 
  \FunctionTok{separate}\NormalTok{(sexage, }\FunctionTok{c}\NormalTok{(}\StringTok{"sex"}\NormalTok{, }\StringTok{"age"}\NormalTok{), }\AttributeTok{sep =} \DecValTok{1}\NormalTok{)}

\NormalTok{who1}
\end{Highlighting}
\end{Shaded}

\begin{Shaded}
\begin{Highlighting}[]
\FunctionTok{library}\NormalTok{(ggplot2) }

\NormalTok{numero\_paises }\OtherTok{=} \DecValTok{16}

\NormalTok{first\_countries }\OtherTok{\textless{}{-}}\NormalTok{ who1 }\SpecialCharTok{\%\textgreater{}\%} 
  \FunctionTok{group\_by}\NormalTok{(country) }\SpecialCharTok{\%\textgreater{}\%} 
    \FunctionTok{summarise}\NormalTok{(}\AttributeTok{sum\_cases =} \FunctionTok{sum}\NormalTok{(cases)) }\SpecialCharTok{\%\textgreater{}\%} 
    \FunctionTok{arrange}\NormalTok{(}\FunctionTok{desc}\NormalTok{(sum\_cases)) }\SpecialCharTok{\%\textgreater{}\%} 
    \FunctionTok{top\_n}\NormalTok{(numero\_paises)}


\NormalTok{who1 }\SpecialCharTok{\%\textgreater{}\%} 
  \FunctionTok{group\_by}\NormalTok{(country,year,sex) }\SpecialCharTok{\%\textgreater{}\%} 
  \FunctionTok{filter}\NormalTok{(country }\SpecialCharTok{\%in\%}\NormalTok{ first\_countries}\SpecialCharTok{$}\NormalTok{country, year}\SpecialCharTok{\textgreater{}}\DecValTok{1995}\NormalTok{) }\SpecialCharTok{\%\textgreater{}\%} 
  \FunctionTok{summarise}\NormalTok{(}\AttributeTok{sum\_cases =} \FunctionTok{sum}\NormalTok{(cases)) }\SpecialCharTok{\%\textgreater{}\%} 
  \FunctionTok{arrange}\NormalTok{(}\FunctionTok{desc}\NormalTok{(sum\_cases)) }\SpecialCharTok{\%\textgreater{}\%} 
  \FunctionTok{ggplot}\NormalTok{(}\AttributeTok{mapping =} \FunctionTok{aes}\NormalTok{(}\AttributeTok{x=}\NormalTok{year,}\AttributeTok{y=}\NormalTok{sum\_cases,}\AttributeTok{colour=}\NormalTok{sex)) }\SpecialCharTok{+}
  \FunctionTok{facet\_wrap}\NormalTok{(}\SpecialCharTok{\textasciitilde{}}\NormalTok{ country, }\AttributeTok{nrow =} \FunctionTok{round}\NormalTok{(}\FunctionTok{sqrt}\NormalTok{(}\FunctionTok{length}\NormalTok{(first\_countries}\SpecialCharTok{$}\NormalTok{country)))) }\SpecialCharTok{+}
  \FunctionTok{ylab}\NormalTok{(}\StringTok{"Número de casos de tuberculosis"}\NormalTok{) }\SpecialCharTok{+}
  \FunctionTok{xlab}\NormalTok{(}\StringTok{"Año"}\NormalTok{) }\SpecialCharTok{+}
  \FunctionTok{geom\_line}\NormalTok{()}
\end{Highlighting}
\end{Shaded}


\end{document}
